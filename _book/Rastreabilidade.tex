% Options for packages loaded elsewhere
\PassOptionsToPackage{unicode}{hyperref}
\PassOptionsToPackage{hyphens}{url}
%
\documentclass[
]{book}
\usepackage{lmodern}
\usepackage{amssymb,amsmath}
\usepackage{ifxetex,ifluatex}
\ifnum 0\ifxetex 1\fi\ifluatex 1\fi=0 % if pdftex
  \usepackage[T1]{fontenc}
  \usepackage[utf8]{inputenc}
  \usepackage{textcomp} % provide euro and other symbols
\else % if luatex or xetex
  \usepackage{unicode-math}
  \defaultfontfeatures{Scale=MatchLowercase}
  \defaultfontfeatures[\rmfamily]{Ligatures=TeX,Scale=1}
\fi
% Use upquote if available, for straight quotes in verbatim environments
\IfFileExists{upquote.sty}{\usepackage{upquote}}{}
\IfFileExists{microtype.sty}{% use microtype if available
  \usepackage[]{microtype}
  \UseMicrotypeSet[protrusion]{basicmath} % disable protrusion for tt fonts
}{}
\makeatletter
\@ifundefined{KOMAClassName}{% if non-KOMA class
  \IfFileExists{parskip.sty}{%
    \usepackage{parskip}
  }{% else
    \setlength{\parindent}{0pt}
    \setlength{\parskip}{6pt plus 2pt minus 1pt}}
}{% if KOMA class
  \KOMAoptions{parskip=half}}
\makeatother
\usepackage{xcolor}
\IfFileExists{xurl.sty}{\usepackage{xurl}}{} % add URL line breaks if available
\IfFileExists{bookmark.sty}{\usepackage{bookmark}}{\usepackage{hyperref}}
\hypersetup{
  pdftitle={Produtos Hortícolas: Rastreabilidade e Requisitos Mínimos},
  pdfauthor={Edição: André Luiz B. Oliveira - Autores:},
  hidelinks,
  pdfcreator={LaTeX via pandoc}}
\urlstyle{same} % disable monospaced font for URLs
\usepackage{longtable,booktabs}
% Correct order of tables after \paragraph or \subparagraph
\usepackage{etoolbox}
\makeatletter
\patchcmd\longtable{\par}{\if@noskipsec\mbox{}\fi\par}{}{}
\makeatother
% Allow footnotes in longtable head/foot
\IfFileExists{footnotehyper.sty}{\usepackage{footnotehyper}}{\usepackage{footnote}}
\makesavenoteenv{longtable}
\usepackage{graphicx,grffile}
\makeatletter
\def\maxwidth{\ifdim\Gin@nat@width>\linewidth\linewidth\else\Gin@nat@width\fi}
\def\maxheight{\ifdim\Gin@nat@height>\textheight\textheight\else\Gin@nat@height\fi}
\makeatother
% Scale images if necessary, so that they will not overflow the page
% margins by default, and it is still possible to overwrite the defaults
% using explicit options in \includegraphics[width, height, ...]{}
\setkeys{Gin}{width=\maxwidth,height=\maxheight,keepaspectratio}
% Set default figure placement to htbp
\makeatletter
\def\fps@figure{htbp}
\makeatother
\setlength{\emergencystretch}{3em} % prevent overfull lines
\providecommand{\tightlist}{%
  \setlength{\itemsep}{0pt}\setlength{\parskip}{0pt}}
\setcounter{secnumdepth}{5}
\usepackage{booktabs}
\usepackage[]{natbib}
\bibliographystyle{apalike}

\title{Produtos Hortícolas: Rastreabilidade e Requisitos Mínimos}
\author{Edição: André Luiz B. Oliveira - Autores:}
\date{2020-11-12}

\begin{document}
\maketitle

{
\setcounter{tocdepth}{1}
\tableofcontents
}
\hypertarget{legislauxe7uxe3o-aplicuxe1vel}{%
\chapter{Legislação Aplicável}\label{legislauxe7uxe3o-aplicuxe1vel}}

Rastreabilidade - A INC nº 02/2018 estabelece os procedimentos para a aplicação da rastreabilidade ao longo da cadeia produtiva de produtos vegetais frescos destinados à alimentação humana, \citep{INC}.

Requisitos Mínimos dos produtos hortícolas - A IN nº 69/2018 estabelece o Regulamento Técnico definindo os requisitos mínimos de identidade e qualidade para Produtos Hortícolas, \citep{REQ}. Essa norma é assistida pelo \href{https://www.gov.br/agricultura/pt-br/assuntos/inspecao/produtos-vegetal/arquivos/referencial-fotografico/referencial-fotografico}{Referencial fotográfico que ilustra os requisitos mínimos}, um conjunto de imagens que facilitam a identificação dos defeitos e condições das frutas, legumes e verduras não conformes com a IN nº 69/2018 e que devem ser removidos do mercado.

Fiscalização dos requisitos mínimos dos produtos hortícolas - A IN nº 07/2019 estabelece os procedimentos simplificados para a fiscalização de produtos hortícolas, \citep{FiscREQ}.

Fundamentação legal - Lei e Decreto que instituem a classificação de produtos vegetais, subprodutos e resíduos de valor econômico, \citep{Lei}, \citep{Decr}.

\hypertarget{intro}{%
\chapter{Introdução}\label{intro}}

\hypertarget{importuxe2ncia-da-fiscalizauxe7uxe3o-como-ferramenta-para-o-cumprimento-da-legislauxe7uxe3o-e-manutenuxe7uxe3o-do-sentimento-de-confianuxe7a-do-consumidor-final-no-abastecimento-e-consumo-de-produtos-de-origem-vegetal-do-brasil}{%
\section{Importância da fiscalização como ferramenta para o cumprimento da legislação e manutenção do sentimento de confiança do consumidor final no abastecimento e consumo de produtos de origem vegetal do Brasil}\label{importuxe2ncia-da-fiscalizauxe7uxe3o-como-ferramenta-para-o-cumprimento-da-legislauxe7uxe3o-e-manutenuxe7uxe3o-do-sentimento-de-confianuxe7a-do-consumidor-final-no-abastecimento-e-consumo-de-produtos-de-origem-vegetal-do-brasil}}

Esta publicação irá tratar principalmente da fiscalização dos produtos hortícolas, conforme definição no Decreto da Classificação Vegetal.

\textbf{Produto hortícola:} produto oriundo da olericultura, da fruticultura, da silvicultura, da floricultura e da jardinocultura.

Iremos detalhar a legislação e os procedimentos de fiscalização a serem adotados para a aplicação das seguintes normas:

\begin{itemize}
\tightlist
\item
  Rastreabilidade, detalhes no Capítulo \ref{rastr};
\item
  Requisitos mínimos, detalhes no Capítulo \ref{req}.
\end{itemize}

Essa publicação foi elaborada em R \textbf{bookdown} \citep{xie2015}.

\hypertarget{rastr}{%
\chapter{Procedimentos de fiscalização de produtos de origem vegetal frescos}\label{rastr}}

Autor
Texto

\hypertarget{detalhamento-da-legislauxe7uxe3o-especuxedfica-para-a-rastreabilidade-atualizada-e-suas-aplicauxe7uxf5es}{%
\section{Detalhamento da Legislação específica para a rastreabilidade atualizada e suas aplicações}\label{detalhamento-da-legislauxe7uxe3o-especuxedfica-para-a-rastreabilidade-atualizada-e-suas-aplicauxe7uxf5es}}

Autor
Texto

\hypertarget{fiscalizauxe7uxe3o-da-rastreabilidade-de-produtos-hortuxedcolas-uxe0-luz-da-instruuxe7uxe3o-normativa-conjunta-nuxba-2-de-07-de-fevereiro-de-2018}{%
\section{Fiscalização da rastreabilidade de produtos hortícolas à luz da Instrução Normativa Conjunta nº 2 de 07 de fevereiro de 2018}\label{fiscalizauxe7uxe3o-da-rastreabilidade-de-produtos-hortuxedcolas-uxe0-luz-da-instruuxe7uxe3o-normativa-conjunta-nuxba-2-de-07-de-fevereiro-de-2018}}

Autor
Texto

\hypertarget{nouxe7uxf5es-buxe1sicas-de-fisiologia-puxf3s-colheita-de-produtos-hortuxedcolas}{%
\chapter{Noções básicas de fisiologia pós colheita de produtos hortícolas}\label{nouxe7uxf5es-buxe1sicas-de-fisiologia-puxf3s-colheita-de-produtos-hortuxedcolas}}

Autor
Texto

\hypertarget{req}{%
\chapter{Procedimentos de fiscalização e amostragem dos produtos hortícolas para avaliação da conformidade aos requisitos mínimos}\label{req}}

\hypertarget{amostragem-para-a-fiscalizauxe7uxe3o-dos-requisitos-muxednimos-dos-produtos-hortuxedcolas}{%
\section{Amostragem para a fiscalização dos requisitos mínimos dos produtos hortícolas}\label{amostragem-para-a-fiscalizauxe7uxe3o-dos-requisitos-muxednimos-dos-produtos-hortuxedcolas}}

Autor
Texto

\hypertarget{fiscalizauxe7uxe3o-dos-requisitos-muxednimos-dos-produtos-hortuxedcolas-uxe0-luz-da-instruuxe7uxe3o-normativa-nuxba-7-de-13-de-maio-de-2019---estabelece-os-procedimentos-simplificados-para-a-fiscalizauxe7uxe3o-de-produtos-hortuxedcolas}{%
\section{Fiscalização dos requisitos mínimos dos produtos hortícolas à luz da instrução normativa nº 7, de 13 de maio de 2019 - Estabelece os procedimentos simplificados para a fiscalização de produtos hortícolas}\label{fiscalizauxe7uxe3o-dos-requisitos-muxednimos-dos-produtos-hortuxedcolas-uxe0-luz-da-instruuxe7uxe3o-normativa-nuxba-7-de-13-de-maio-de-2019---estabelece-os-procedimentos-simplificados-para-a-fiscalizauxe7uxe3o-de-produtos-hortuxedcolas}}

Autor
Texto

\hypertarget{utilizauxe7uxe3o-do-referencial-fotogruxe1fico-dos-produtos-hortuxedcolas-durante-a-fiscalizauxe7uxe3o}{%
\section{Utilização do Referencial fotográfico dos produtos hortícolas durante a fiscalização}\label{utilizauxe7uxe3o-do-referencial-fotogruxe1fico-dos-produtos-hortuxedcolas-durante-a-fiscalizauxe7uxe3o}}

Autor
Texto
\href{https://www.gov.br/agricultura/pt-br/assuntos/inspecao/produtos-vegetal/arquivos/referencial-fotografico/referencial-fotografico}{Link à página do MAPA com o referencial fotográfico dos produtos hortícolas}

\hypertarget{considerauxe7uxf5es-finais}{%
\chapter{Considerações finais}\label{considerauxe7uxf5es-finais}}

  \bibliography{book.bib,packages.bib}

\end{document}
